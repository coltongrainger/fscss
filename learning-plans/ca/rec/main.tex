\documentclass[10pt]{letter}

\usepackage{geometry}
\usepackage{wallpaper}
\geometry{
	top=1.2in,
	bottom=0.1in,
	right=1.25in,
	left=1.25in
}

\ULCornerWallPaper{1}{fscss-letterhead.pdf}

\usepackage{fontspec}
\setmainfont{TeX Gyre Schola}

\usepackage[super]{nth}
\usepackage{hyperref}

\title{connie-rec-letter}

\begin{document}

\begin{letter}{
	Across the Pond\\
	170 S. Green Valley Parkway\\
	Henderson, NV 89012
}
\opening{To Whom It May Concern,}

``Connie'' Adamson has volunteered with our organization, at a shelter for families experiencing homelessness, since September 2017. From now until March 2018, Connie is expected to intern 5 hours each week within our organization. Her expressed goal is to understand ``issues [clients face] within the public system'' of housing.

I recommend Connie for graduate study on the basis of her motivation to research how populations create and maintain their individual and collective identities. 
	
In her internship, Connie plans (i)~to conduct interviews with our clients, volunteers, and staff, and (ii)~situate these interviews within global perspectives of homelessness. I believe Connie will succeed in characterizing the historical and geographic antecedents to our staff and clients' beliefs about themselves. With these characterizations, I expect Connie to analyze both how families see themselves and how the community ``narrates'' their experiences.

	Connie is an informed volunteer, eager to relate theory to practice. I believe only 1 in 20 folks in social services share Connie's philosophical disposition. To challenge Connie, I have asked Connie to survey zine literature on radical parenting\footnote{C. Martens, \emph{The Future Generation}. Atomic Book Company: Baltimore, MD (2007).} I hope that Connie proceeds to challenge the stigma that social reproduction is inherently unpunk or unacademic. I believe Connie would benefit from seeing how critical arguments can be applied for a community's welfare. I have often joked with Connie that, working with survivors of trauma, cultivating child-like spoantaniety is ``applied philosophy.''

	In terms of skills relevant to research in Anthropology, while working at our shelter Connie has learned:
	how to build repoire with community members;
	how to de-escalate conflicts while remaining impartial; 
	how to engage with families of various demographic backgrounds; and
	how private foundations, non-profits, and public agencies provide a social safety-net in the Northwest United States. 

	I believe Connie's insight from volunteering and interning at the Family Support Center prepares her formidably for graduate study in Anthropology. I reiterate that Connie would benefit from seeing how critical theory can be applied for a community's welfare. Given her interdisciplinary background, I have no doubt that she will exert an enthusiastic and sustained effort in her graduate work.
	
\closing{
	Respectfully,\\
	\fromsig{\includegraphics[scale=1]{signature.png}}\\
	\fromname{
		Colton Grainger\\
		Community Engagement Specialist\\
		\href{mailto:coltong@fscss.org}{\texttt{coltong@fscss.org}}
	}
}

\end{letter}
\end{document}


