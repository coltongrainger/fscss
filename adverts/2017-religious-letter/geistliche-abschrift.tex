\documentclass{letter}

\usepackage[margin=1in]{geometry}
\usepackage{textmerg}

\usepackage{tgbonum}

\signature{Dagmar Growe}

\begin{document}

\Fields{\Fullname\Address}

\Merge{oly-churches.dat}{
    \begin{letter}{
    	\Fullname\\
	\Address
    }
    \opening{
	Dear Sisters and Brothers in Christ,
    }

My name is Dagmar Growe, and I am a volunteer with the Family Support
Center. I am writing to draw your attention to a desperate volunteer
need in our community.

The Family Support Center operates the only shelter for homeless
families in the county. Without Pear Blossom Place, somewhere between 20 
and 30 children would have to spend the night sleeping outside or in
cars each night. As most non-profit social service agencies, the Family
Support Center operates on a shoe-string budget, and does not have
funding for overnight and weekend staff. These have to be covered by
volunteers.

In the past, many of those shifts have been covered by work-study
students. For a number of reasons, this is not happening anymore, and
the lack of volunteers is creating a critical and desperate situation.
Staff members have had to cover night shifts, which means they are less
available during the day to provide much needed case management and
other services which are designed to move families into permanent
housing and keep them there. The fact that staff members are willing to
give up their weekend time and nights multiple times a week speaks more
then anything to their dedication to the families, but it is not
sustainable situation, and will, I am sure, lead to burn-out, attrition,
and further instability in the lives of the families they serve.
Legally, the shelter is not allowed to operate without continuous
staffing.

Volunteer shifts are not hard---as an overnight volunteer I have a
choice between sleeping in a loft bed or on a cot in one of the staff
offices. Families live in ``apartments'' off the main area, so after 9 pm
things get pretty quiet. I usually get a good night's sleep and I work
the next day. My ``work'' at the shelter consists of controlling access
to the building via an intercom, answering the phone (rarely), and hanging
out with guests as much as I feel ready to engage. Especially the kids
tend to be eager for attention, but parents too can be glad for someone
to listen. I don't admit families to the shelter---this is done only by
staff---and there is always a staff person on call in case of problems
(although I can't remember the last time I called).

Coverage is needed from 5pm to 7am on week nights and shifts are split
between 2 greeters in the early evening, and 2 hosts who come in later
and spend the night. Weekends require additional day-time coverage.
Training is provided, and background checks are required, so there would
need to be some regular commitment. If your congregation could assemble
a team of 6--8 people and commit to covering 2 nights each month it
would make a huge difference. However, I do find that my volunteering
becomes more rewarding when I volunteer weekly, as I get to know the
families a little better.

For me, spending the night at the shelter has become rewarding way to
put my faith into action and follow Jesus' call to serve. I am asking
you to share this opportunity with your congregation and prayerfully
consider your response. Please contact Colton, 360-754-9297 ext.~306, \texttt{coltong@fscss.org}, at the Family Support Center for more information.

\closing{Blessings,}
\end{letter}
}

\end{document}
